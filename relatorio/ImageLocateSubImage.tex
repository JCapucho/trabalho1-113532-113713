\renewcommand{\listingscaption}{Caso em estudo:}

\chapter[ImageLocateSubImage]{Análise da Função ImageLocateSubImage}
A primeira das funções a analisar é a ImageLocateSubImage. Iremos analisar o número de comparações efetuadas envolvendo
o valor de cinzento entre pixeis das imagens. Para começar, iremos realizar uma análise informal, realizando vários
testes e registando o número de comparações. De seguida, iremos analisar formalmente o algorítmo utilizado e comparar
com o resultado obtido pela análise informal.

\section[Análise informal]{Análise Informal da Função ImageLocateSubImage
  \label{informal_locate}}
Para a realização de uma análise informal, é necessário que a cada utilização da função registemos o número de
comparações realizadas entre valores de cinzento entre pixeis. Para tal podemos guardar valores no array global
InstrCount. No final da execução, temos que imprimir para a consola ou para um ficheiro o valor guardado.

Podemos então depois de executar alguns testes, comparar resultados. Consideremos então os testes realizados com 2
imagens (1 e 2). Nos testes, iremos verificar se a imagem 1
contém a imagem 2. Consideremos então \verb|H| e \verb|W|
como sendo a altura e
largura da imagem 1 (respetivamente), e \verb|h| e \verb|w|
como sendo a altura e largura da figura 2.

Observemos os seguintes resultados:
\begin{listing}[H]
	\centering
	Image1: 20 x 40;	Image2: 2 x 16;	Number of grey comparisons: 15200;

	Image1: 20 x 40;	Image2: 16 x 2;	Number of grey comparisons: 6240;
	\caption{Resultados para imagens de igual área e resolução
		diferente}
	\label{eq_area_diff_res}
\end{listing}

Por estes resultados, podemos concluir que o número de
comparações efetuadas não depende da área das imagens, uma
vez que no \textbf{caso em estudo \ref{eq_area_diff_res}} as
imagens têm a mesma área, mas número de comparações efetuadas
diferente. Deste modo, a função complexidade dependerá sempre das
largura e altura de ambas as imagens e não apenas do número
de píxeis em cada imagem.

\section{Análise Formal da Função ImageMatchSubImage}
\renewcommand{\listingscaption}{Código:}
Uma vez que a nossa função ImageLocateSubImage chama a
função ImageMatchSubImage, comecemos por analisar esta.

A função consiste na realização de dois \verb|for loops|
encadeados:
\begin{listing}[H]
	\centering
	\begin{minted}{c}
int ImageMatchSubImage(Image img1, int x, int y, Image img2) { ///
  // (...)
  for (int new_x = 0; new_x < img2->width; new_x++) {
    for (int new_y = 0; new_y < img2->height; new_y++) {
      const int img1_value = ImageGetPixel(img1, new_x + x, new_y + y);
      const int img2_value = ImageGetPixel(img2, new_x, new_y);
      GREYCMP++;
    }
  }
  return 1;
\end{minted}
	\caption{Implementação da função ImageMatchSubImage}
\end{listing}

O pior caso da execução pode ser traduzida matemáticamente como
\begin{equation}
	\sum_{a=0}^{w}\sum_{b=0}^{h}1 = w \cdot h
\end{equation}

O melhor caso será aquele em que o 1º pixel da imagem
1 é diferente do 1º pixel da imagem 2, que resultará num
retorno imediato da função. Como tal o melhor caso é tempo
constante.
Como tal a análise formal da função resulta em $B(w,h) = 1$
e $W(w,h) = w \cdot h$, sendo w e h a largura e altura da
imagem mais pequena.


\section{Análise Formal da Função ImageLocateSubImage}

Para a análise formal da função em estudo, podemos apenas
apoiar-nos na análise já efetuada para a função
ImageMatchSubImage.

ImageLocateSubImage consiste também ela num encadeamento de
\verb|for-loops|, pelo que a análise efetuada será
semelhante.

\begin{listing}[H]
	\centering
	\begin{minted}{c}
int ImageLocateSubImage(Image img1, int *px, int *py, Image img2) {
    // (...)
  for (int x = 0; x <= img1->width - img2->width; x++) {
    for (int y = 0; y <= img1->height - img2->height; y++) {
      if (ImageMatchSubImage(img1, x, y, img2)) {
        *px = x;
        *py = y;
        return 1;
      }
    }
  }
  return 0;
}
  \end{minted}
\end{listing}

Uma vez mais, o pior caso será a realização de ambos os
\verb|for loops| na sua totalidade. Como tal, podemos
uma vez mais traduzir matemáticamente a expressão:

\begin{equation}
	\sum_{a=0}^{W-w+1}\sum_{b=0}^{H-h+1}w\cdot h = (W-w+1)\cdot(H-h+1)\cdot(w\cdot h)
\end{equation}

Importante notar que nesta função, o seu melhor caso será
aquele em que a função ImageMatchSubImage retorne 1, ou
seja, o pior caso para a função ImageMatchSubImage, que já
calculamos como sendo $w \cdot h$

Como tal, temos que $B(W,H,w,h) = w\cdot h$ e $W(W,H,w,h) = (W-w+1)\cdot(H-h+1)\cdot(w\cdot h)$

Para a análise do caso médio, temos agora que ter em consideração o \verb|maxval|, ou seja, o valor máximo da
intensidade de um pixel na imagem. Se considerarmos imagens aleatórias onde a intensidade de cada píxel é equiprovável,
então a probabilidade de 2 píxeis serem iguais é
$\frac{1}{maxval + 1}$
Como a probabilidade de n píxeis seguidos serem iguais corresponde
a $(\frac{1}{maxval + 1})^n$, este valor rápidamente tende
para 0, pelo que podemos concluir que o número de
comparações efetuado será relativamente pequeno. Se
considerarmos que temos que comparar 10\% dos pixeis, então
o caso médio do ImageMatch ficará $A(w,h)= \frac{w
\cdot h}{10}$. Para o caso médio do ImageLocateSubImage, podemos
assumir que a sub-imagem é encontrada aproximadamente a
meio, sendo entao $A(W,H,w,h)=(W-w)\cdot\frac{(H-h)}{2} \cdot
\frac{w\cdot h}{10}$

\section[Resultados]{Comparação do resultado obtido em ambas as análises}

Através dos resultados obtidos em \ref{informal_locate},
concluímos que o número de comparações era dependente da
resolução de cada imagem e não somente do seu número de píxeis. Após a análise formal podemos concluir que
essa observação é válida.
