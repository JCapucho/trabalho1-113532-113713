\renewcommand{\listingscaption}{Caso em estudo:}

\chapter{Análise da Função \VerbSection{ImageLocateSubImage}}

A primeira função a analisar é a \Verb|ImageLocateSubImage|, mais
especificamente, o número de comparações efetuadas entre os valores de
cinzento dos pixeis das duas imagens. Para começar, iremos realizar uma análise
informal, realizando vários testes e registando o número de comparações.
De seguida, iremos analisar formalmente o algoritmo utilizado.

\section{Análise Informal da Função \VerbSection{ImageLocateSubImage}}

Para a realização da análise informal procuramos, para cada invocação
da função, obter o número de comparações realizadas entre os valores de
cinzento dos pixeis. Para tal utilizamos as ferramentas de instrumentação que
nos foram disponibilizadas, sendo destacados os comandos \Verb|tic| e \Verb|toc|.

Vamos então considerar testes realizados com 2 imagens (1 e 2). Nos testes,
iremos verificar se a imagem 1 contém a imagem 2.

Dos resultados obtidos destacam-se os seguintes:

\begin{listing}[H]
	\centering
	\begin{minted}{text}
	Image1: 20 x 40; Image2: 2 x 16; Number of grey comparisons: 15200;
	Image1: 20 x 40; Image2: 16 x 2; Number of grey comparisons: 6240;
  \end{minted}
	\caption{Resultados para imagens de igual área e resolução diferente}
\end{listing}

Estes permitem-nos concluir que o número de comparações efetuadas não depende só
da área das imagens, uma vez que as imagens têm a mesma área, mas o número de
comparações efetuadas é diferente. Deste modo, a função complexidade dependerá
da largura e altura das imagens e não apenas do número de pixeis de cada imagem.

\section{Análise Formal da Função \VerbSection{ImageMatchSubImage}}
\renewcommand{\listingscaption}{Código:}
A função \Verb|ImageLocateSubImage|, que implementamos, invoca a função
\Verb|ImageMatchSubImage|, por isso vamos começar por fazer uma breve análise
desta.

\begin{listing}[H]
	\centering
	\begin{minted}{c}
  FOR_COORDINATES(img2, x, y) {
    if (img1_value != img2_value)
      return 0;
  }
  return 1;
  \end{minted}
	\caption{Implementação da função ImageMatchSubImage}
	\label{code:locate:ImageMatchSubImage}
\end{listing}

Como podemos ver pelo código \ref{code:locate:ImageMatchSubImage}, a função
consiste na iteração da imagem que estamos a procura, onde em cada iteração
comparamos com o pixel correspondente da imagem onde estamos a procurar.

Consideremos \verb|H| e \verb|W| como sendo a altura e largura, respetivamente,
da imagem onde vamos procurar, e \verb|h| e \verb|w| como sendo a altura e
largura da imagem que estamos a procura. Podemos então traduzir o pior caso da
execução na seguinte expressão matemática:

\begin{equation}
	W(w,h) = \sum_{x = 0}^{w - 1} \sum_{y = 0}^{h -1} 1 = w \cdot h
\end{equation}

O melhor caso será aquele em que o 1º pixel da imagem onde estamos a procurar é
diferente do 1º pixel da imagem que estamos a procura, pois isto resulta num
retorno imediato da função. Como tal o melhor caso é $B(w,h) = 1$.


\section{Análise Formal da Função \VerbSection{ImageLocateSubImage}}

\Verb|ImageLocateSubImage| consiste também ela na iteração da imagem, mas desta
vez apenas pela diferença de tamanhos das duas imagens.

\begin{listing}[H]
	\centering
	\begin{minted}{c}
  const int check_width = img1->width - img2->width;
  const int check_height = img1->height - img2->height;
  FOR_COORDINATES_SIZED(x, y, check_width, check_height) {
    if (ImageMatchSubImage(img1, x, y, img2))
      return 1;
  }
  return 0;
  \end{minted}
\end{listing}

Uma vez mais, o pior caso será a iteração de todos os pixeis na sua totalidade,
obtendo assim a seguinte expressão:

\begin{equation}
	W(W,H,w,h) = \sum_{x = 0}^{W-w+1} \, \sum_{y = 0}^{H-h+1}w\cdot h = (W-w+1)\cdot(H-h+1)\cdot(w\cdot h)
\end{equation}

Importante realçar que nesta função, o melhor caso varia de acordo com a
diferença entre as duas imagens. Se $(W - w) \cdot (H - h) < w \cdot h$ o melhor
caso é quando todas as comparações falham no primeiro pixel, logo
$B(W,H,w,h) = (W - w) \cdot (H - h)$.
Caso contrário, o melhor caso implica que a imagem seja encontrada no canto
superior direito, o que corresponde ao pior caso para a função
\Verb|ImageMatchSubImage|, que já calculamos como sendo $w \cdot h$. Como tal,
temos que $B(W,H,w,h) = w\cdot h$.

Para a análise do caso médio, temos agora que ter em consideração o valor máximo
de intensidade de um pixel na imagem, $maxval$.
Se considerarmos imagens aleatórias onde a intensidade de cada pixel é
equiprovável, então a probabilidade de 2 pixeis serem iguais é
$\frac{1}{maxval + 1}$.
Como a probabilidade de n pixeis seguidos serem iguais corresponde a
$(\frac{1}{maxval + 1})^n$, este valor rapidamente tende para 0.

Se considerarmos que temos que comparar 10\% dos pixeis, então o caso médio do
\Verb|ImageMatchSubImage| ficará $A(w,h)= \frac{w \cdot h}{10}$.
Para o caso médio do \Verb|ImageLocateSubImage|, podemos assumir que a
sub-imagem é encontrada aproximadamente a meio, sendo então
$A(W,H,w,h)=(W-w)\cdot\frac{(H-h)}{2} \cdot \frac{w\cdot h}{10}$.
