\chapter{Análise da função \VerbSection{ImageBlur}}

\section{Primeira Implementação}

No código abaixo é apresentada uma versão simplificada da primeira implementação
que apenas mostra onde \Verb|PIXMEM| é atualizado.

\begin{listing}[H]
	\centering
	\begin{minted}{c}
  for (int x = 0; x < img->width; x++) {
    for (int y = 0; y < img->height; y++) {
      for (int win_x = -dx; win_x <= dx; win_x++) {
        for (int win_y = -dy; win_y <= dy; win_y++) {
          // Soma de cada pixel na janela do filtro blur
          PIXMEM++;
        }
      }
      // Cálculo e armazenamento do valor com blur aplicado
      PIXMEM++;
    }
  }
  \end{minted}
\end{listing}

Consideremos $w$ e $h$ como sendo a largura e a altura da imagem, e $dx$
o raio horizontal e $dy$ o raio vertical da janela de blur.

O número de acessos realizados por esta implementação pode ser representado
matematicamente pela seguinte expressão:

\begin{align}
	  & \sum_{x = 0}^{w - 1} \sum_{y = 0}^{h - 1} \left(
	1 +
	\sum_{x_{win} = -dx}^{dx} \,
	\sum_{y_{win} = -dy}^{dy} 1
	\right)                                                       \\
	= & \sum_{x = 0}^{w - 1} \sum_{y = 0}^{h - 1} 1 +
	\sum_{x = 0}^{w - 1} \sum_{y = 0}^{h - 1} \,
	\sum_{x_{win} = -dx}^{dx} \, \sum_{y_{win} = -dy}^{dy} 1      \\
	= & (w \cdot h) + (w \cdot h \cdot (2dx + 1) \cdot (2dy + 1))
	\label{eq:blur:first_implementation}
\end{align}

A expressão que obtivemos não depende dos valores dos pixeis, nem de outros
fatores além das dimensões da imagem e da janela do filtro, logo vamos ter que o
melhor caso, o caso médio e o pior caso serão todos iguais.

Analisando a expressão \eqref{eq:blur:first_implementation}, o segundo termo
desta cresce a um ritmo superior do que o outro termo presente na mesma, logo
podemos deduzir que a ordem de complexidade associada a este algoritmo é de
$\mathcal{O}(w \cdot h \cdot 2dx \cdot 2dy)$.

\section{Segunda Implementação}

A visão simplificada da segunda implementação é apresentada no código seguinte.

\begin{listing}[H]
	\centering
	\begin{minted}{c}
  // Inicialização vetor soma
  for (int x = 0; x < img->width; x++) {
    PIXMEM++;
    for (int win_y = 1; win_y < radius_y; win_y++) {
      PIXMEM++;
    }
    PIXMEM++;
  }

  for (int y = 0; y < img->height; y++) {
    if (y != 0) {
      // Atualização do vetor soma
      for (int x = 0; x < img->width; x++) {
        PIXMEM += 2;
      }
    }

    // Cálculo do blur para o pixel (0, y)
    PIXMEM++;

    for (int x = 1; x < img->width; x++) {
      // Cálculo do blur para o pixel (x, y)
      PIXMEM++;
    }
  }
  \end{minted}
\end{listing}

\Verb|radius_x| é igual a $dx$ ou $w$, o que for menor (o mesmo acontece para
\Verb|radius_y| com $dy$ e $h$). Vamos assumir na nossa análise que $dx < w$ e
$dy < h$.

Logo obtemos a seguinte expressão:

\begin{align}
	  & \sum_{x = 0}^{w - 1} \left[
		1
		+ \left(\sum_{y_{win} = 1}^{dy - 1} 1\right)
		+ 1
		\right]
	+ \sum_{y = 0}^{h - 1} \left( 1 + \sum_{x = 1}^{w - 1} 1 \right)
	+ \sum_{y = 1}^{h - 1} \sum_{x = 0}^{w - 1} 2                  \\
	= & \left( w + w \cdot (dy - 1) + w \right)
	+ \left( h + h \cdot (w - 1) \right)
	+ 2(h - 1) \cdot w                                             \\
	= & w \cdot (dy + 1) + h \cdot w + 2(h - 1) \cdot w            \\
	= & w \cdot dy + w + h \cdot w + 2 \cdot h \cdot w - 2 \cdot w \\
	= & 3 \cdot h \cdot w + w \cdot dy - w
	\label{eq:blur:second_implementation}
\end{align}

Mais uma vez temos que o melhor caso, o caso médio e o pior caso são iguais

E observando a expressão \eqref{eq:blur:second_implementation}, o termo
$3 \cdot h \cdot w$ cresce mais rapidamente que os restantes a medida que os
parâmetros aumentam, logo podemos concluir que a ordem de complexidade é de
$\mathcal{O}(3 \cdot h \cdot w)$.

\section{Comparação de resultados}

\begin{table}[H]
	\centering
	\begin{tabular}{ |l||c|c||c|c||c|c|  }
		\hline
		\multirow{2}{*}{Versão}                 & \multicolumn{2}{c||}{Tamanho imagem} & \multicolumn{2}{c||}{Raio do filtro} & \multirow{2}{*}{Pixeis lidos} & \multirow{2}{*}{Divisões}                    \\
		\cline{2-5}
		                                        & largura                              & altura                               & horizontal                    & vertical                  &          &       \\
		\hline
		\multirow{6}{*}{Primeira implementação} & 8                                    & 8                                    & 1                             & 1                         & 640      & 64    \\
		\cline{2-7}
		                                        & 8                                    & 8                                    & 3                             & 3                         & 3200     & 64    \\
		\cline{2-7}
		                                        & 100                                  & 100                                  & 5                             & 5                         & 1220000  & 10000 \\
		\cline{2-7}
		                                        & 100                                  & 100                                  & 10                            & 10                        & 4420000  & 10000 \\
		\cline{2-7}
		                                        & 300                                  & 300                                  & 7                             & 7                         & 20340000 & 90000 \\
		\cline{2-7}
		                                        & 300                                  & 300                                  & 15                            & 15                        & 86580000 & 90000 \\
		\hline
		\multirow{6}{*}{Segunda implementação}  & 8                                    & 8                                    & 1                             & 1                         & 192      & 64    \\
		\cline{2-7}
		                                        & 8                                    & 8                                    & 3                             & 3                         & 208      & 64    \\
		\cline{2-7}
		                                        & 100                                  & 100                                  & 5                             & 5                         & 30400    & 10000 \\
		\cline{2-7}
		                                        & 100                                  & 100                                  & 10                            & 10                        & 30900    & 10000 \\
		\cline{2-7}
		                                        & 300                                  & 300                                  & 7                             & 7                         & 271800   & 90000 \\
		\cline{2-7}
		                                        & 300                                  & 300                                  & 15                            & 15                        & 274200   & 90000 \\
		\hline
	\end{tabular}
\end{table}
