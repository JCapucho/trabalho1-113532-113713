\chapter[ImageBlur]{Análise da função ImageBlur}

\section{Primeira Implementação}
Consideremos uma versão bastante simplificada da função em que apenas consideramos os incrementos a PIXMEM. Consideremos
\verb|w| e \verb|h| como sendo largura e altura da imagem,
\verb|dx| o raio horizontal da janela de blur e \verb|dy| o
raio vertical da janela de blur;
\begin{listing}[H]
	\centering
	\begin{minted}{c}
  // (...) 
  for (int x = 0; x < img->width; x++) {
    for (int y = 0; y < img->height; y++) {
      for (int win_x = -dx; win_x <= dx; win_x++) {
        for (int win_y = -dy; win_y <= dy; win_y++) {
          PIXMEM++;
        }
      }
      PIXMEM++;
    }
  }
  // (...)
\end{minted}
\end{listing}

Matemáticamente representável como:
\\
\begin{equation}
	\sum_{a=0}^{w}\sum_{b=0}^{h} (1 + \sum_{c=-dx}^{dx+1}\sum_{d=-dy}^{dy+1} 1)
	= \sum_{a=0}^{w}\sum_{b=0}^{h} 1 +
	\sum_{a=0}^{w}\sum_{b=0}^{h}
	\sum_{c=-dx}^{dx+1}\sum_{d=-dy}^{dy+1} 1
\end{equation}
\begin{align*}
	= (w \cdot h) + (w \cdot h \cdot (2dx + 1) \cdot (2dy + 1))
\end{align*}

Como o segundo aditivo admite valores muito superiores ao
primeiro, podemos apenas não considerar o primeiro.

Então obtemos $w\cdot h \cdot (2dx + 1) \cdot (2dy + 1)$
Se considerarmos \verb|x| e \verb|y| como sendo as dimensões totaisda janela de blur, então $2dx + 1 = x$ e $2dy + 1 = y$,
pelo que a nossa primeira implementação de ImageBlur obtém
uma complexidade computacional de $\mathcal{O}(w\cdot h\cdot
	x\cdot y)$, sendo w e h as dimensões da imagem a desfocar e
x e y o tamanho da janela utilizada.



\section{Segunda Implementação}
